%%%%%%%%%%%%%%%%%%%%%%%%%%%%%%%%%%%%%%%%%%%
%%% DOCUMENT PREAMBLE %%%
\documentclass[12pt]{report}
\usepackage[english]{babel}
%\usepackage{natbib}
\usepackage{url}
\usepackage[utf8x]{inputenc}
\usepackage{amsmath}
\usepackage{graphicx}
\graphicspath{{images/}}
\usepackage{parskip}
\usepackage{fancyhdr}
\usepackage{vmargin}
\usepackage{listings}
\usepackage{hyperref}
\usepackage{xcolor}

\definecolor{codegreen}{rgb}{0,0.6,0}
\definecolor{codegray}{rgb}{0.5,0.5,0.5}
\definecolor{codeblue}{rgb}{0,0,0.95}
\definecolor{backcolour}{rgb}{0.95,0.95,0.92}

\lstdefinestyle{mystyle}{
    backgroundcolor=\color{backcolour},   
    commentstyle=\color{codegreen},
    keywordstyle=\color{codeblue},
    numberstyle=\tiny\color{codegray},
    stringstyle=\color{codegreen},
    basicstyle=\ttfamily\footnotesize,
    breakatwhitespace=false,         
    breaklines=true,                 
    captionpos=b,                    
    keepspaces=true,                 
    numbers=left,                    
    numbersep=5pt,                  
    showspaces=false,                
    showstringspaces=false,
    showtabs=false,                  
    tabsize=2
}
 
\lstset{style=mystyle}

\setmarginsrb{3 cm}{2.5 cm}{3 cm}{2.5 cm}{1 cm}{1.5 cm}{1 cm}{1.5 cm}

\title{Lab 5}								
% Title
\author{ Abdalrahman Alhajri}						
% Author
\date{Oct 6, 2021}
% Date

\makeatletter
\let\thetitle\@title
\let\theauthor\@author
\let\thedate\@date
\makeatother

\pagestyle{fancy}
\fancyhf{}
\rhead{\theauthor}
\lhead{\thetitle}
\cfoot{\thepage}
%%%%%%%%%%%%%%%%%%%%%%%%%%%%%%%%%%%%%%%%%%%%
\begin{document}

%%%%%%%%%%%%%%%%%%%%%%%%%%%%%%%%%%%%%%%%%%%%%%%%%%%%%%%%%%%%%%%%%%%%%%%%%%%%%%%%%%%%%%%%%

\begin{titlepage}
	\centering
    \vspace*{0.5 cm}
   % \includegraphics[scale = 0.075]{bsulogo.png}\\[1.0 cm]	% University Logo
\begin{center}    \textsc{\Large   ECE 351 - Section \#52 }\\[2.0 cm]	\end{center}% University Name
	\textsc{\Large Lab 5 Report }\\[0.5 cm]				% Course Code
	\rule{\linewidth}{0.2 mm} \\[0.4 cm]
	{ \huge \bfseries \thetitle}\\
	\rule{\linewidth}{0.2 mm} \\[1.5 cm]
	
	\begin{minipage}{0.4\textwidth}
		\begin{flushleft} \large
		%	\emph{Submitted To:}\\
		%	Name\\
          % Affiliation\\
           %contact info\\
			\end{flushleft}
			\end{minipage}~
			\begin{minipage}{0.4\textwidth}
            
			\begin{flushright} \large
			\emph{Submitted By :} \\
			Abdalrahman Alhajri  
		\end{flushright}
           
	\end{minipage}\\[2 cm]
	
%	\includegraphics[scale = 0.5]{PICMathLogo.png}
    
    
    
    
	
\end{titlepage}

%%%%%%%%%%%%%%%%%%%%%%%%%%%%%%%%%%%%%%%%%%%%%%%%%%%%%%%%%%%%%%%%%%%%%%%%%%%%%%%%%%%%%%%%%

\tableofcontents
\pagebreak

%%%%%%%%%%%%%%%%%%%%%%%%%%%%%%%%%%%%%%%%%%%%%%%%%%%%%%%%%%%%%%%%%%%%%%%%%%%%%%%%%%%%%%%%%
\renewcommand{\thesection}{\arabic{section}}
\section{Introduction}
The goal in this lab is to use Laplace transforms to find the time-domain response of an RLC bandpass filter to impulse and step inputs. In order to do so, we did the pre-lab by symbolically finding the transfer function $H(s)$ for the bandpass filter and using it to find the impulse response $h(t)$.

\section{Equations}
In the second part of the lab, the final value theorem is asked to be performed for the step response $H(s)u(s)$

$$\lim_{t\to \infty} h(t)u(t) = \lim_{s\to 0} sH(s)u(s)= \lim_{s\to 0} s\cdot \frac{\frac{1}{RC}s}{s^2+\frac{1}{RC}s+\frac{1}{LC}} \cdot \frac{1}{s} = 0 $$
This agrees with figure 2, we see that as t goes to infinity, the step response reaches zero.

\section{Methodology}

This lab consists of two parts:
\begin{itemize}
    \item
    Part 1:\\
    In this part, we defined a function for the bandpass filter taking the values of the following inputs; R, L, C, t. Then we plotted the function for $0 ≤ t ≤ 1.2 ms$. after that we compared it with the scipy.signal.impulse() function. Both plots are shown in figure 1.

    \item
    Part 2:\\
    This part focuses on the step response of the transform function and how it behaves. To do so, we used the built-in scipy.signal.step() function for our filter and plotted it as shown in figure 2.\\
    In addition, we applied the final value theorem on the step response as shown in the Equation section. It shows that when performing the final value theorem for the step response, we are in fact evaluating the impulse response of the transform function as $s\to 0$. 
\end{itemize}
   
\section{Results}
\begin{itemize}
    \item Part 1:
    The two plots for the user-defined and built-in functions of the impulse response are shown in figure 1 as following:\\
    \includegraphics[]{figure 1.png}\\
    
     \item Part 2:
    The built-in step response function for the given bandpass filter is plotted in figure 2 as following:\\
    \includegraphics[]{figure 2.png}
    
\end{itemize}


\section{Questions}

\begin{enumerate}
    \item 
    Explain the result of the Final Value Theorem from Part 2 in terms of the physical circuit components.\\
    we applied the final value theorem on the step response as shown in the Equation section. It shows that when performing the final value theorem for the step response, we are in fact evaluating the impulse response of the transform function as $s\to 0$.
    $$\lim_{t\to \infty} h(t)u(t) = \lim_{s\to 0} sH(s)u(s)= \lim_{s\to 0} s\cdot \frac{\frac{1}{RC}s}{s^2+\frac{1}{RC}s+\frac{1}{LC}} \cdot \frac{1}{s} = 0 $$
    \item
    Leave any feedback on the clarity of the expectations, instructions, and deliverables.\\
     Thank you.
\end{enumerate}

\section{Conclusion}
Throughout this lab, we built our own user-defined impulse response function for a bandpass filter and learned how it functions. We also took a look at how the step response behave and that when we apply the final value theorem to it we get the value of the impulse response as t goes to infinity.

\end{document}

%This template was created by Roza Aceska.