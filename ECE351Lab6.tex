%%%%%%%%%%%%%%%%%%%%%%%%%%%%%%%%%%%%%%%%%%%
%%% DOCUMENT PREAMBLE %%%
\documentclass[12pt]{report}
\usepackage[english]{babel}
%\usepackage{natbib}
\usepackage{url}
\usepackage[utf8x]{inputenc}
\usepackage{amsmath}
\usepackage{graphicx}
\graphicspath{{images/}}
\usepackage{parskip}
\usepackage{fancyhdr}
\usepackage{vmargin}
\usepackage{listings}
\usepackage{hyperref}
\usepackage{xcolor}

\definecolor{codegreen}{rgb}{0,0.6,0}
\definecolor{codegray}{rgb}{0.5,0.5,0.5}
\definecolor{codeblue}{rgb}{0,0,0.95}
\definecolor{backcolour}{rgb}{0.95,0.95,0.92}

\lstdefinestyle{mystyle}{
    backgroundcolor=\color{backcolour},   
    commentstyle=\color{codegreen},
    keywordstyle=\color{codeblue},
    numberstyle=\tiny\color{codegray},
    stringstyle=\color{codegreen},
    basicstyle=\ttfamily\footnotesize,
    breakatwhitespace=false,         
    breaklines=true,                 
    captionpos=b,                    
    keepspaces=true,                 
    numbers=left,                    
    numbersep=5pt,                  
    showspaces=false,                
    showstringspaces=false,
    showtabs=false,                  
    tabsize=2
}
 
\lstset{style=mystyle}

\setmarginsrb{3 cm}{2.5 cm}{3 cm}{2.5 cm}{1 cm}{1.5 cm}{1 cm}{1.5 cm}

\title{Lab 6: Partial Fraction Expansion}								
% Title
\author{ Abdalrahman Alhajri}						
% Author
\date{Oct 13, 2021}
% Date

\makeatletter
\let\thetitle\@title
\let\theauthor\@author
\let\thedate\@date
\makeatother

\pagestyle{fancy}
\fancyhf{}
\rhead{\theauthor}
\lhead{\thetitle}
\cfoot{\thepage}
%%%%%%%%%%%%%%%%%%%%%%%%%%%%%%%%%%%%%%%%%%%%
\begin{document}

%%%%%%%%%%%%%%%%%%%%%%%%%%%%%%%%%%%%%%%%%%%%%%%%%%%%%%%%%%%%%%%%%%%%%%%%%%%%%%%%%%%%%%%%%

\begin{titlepage}
	\centering
    \vspace*{0.5 cm}
   % \includegraphics[scale = 0.075]{bsulogo.png}\\[1.0 cm]	% University Logo
\begin{center}    \textsc{\Large   ECE 351 - Section \#52 }\\[2.0 cm]	\end{center}% University Name
	\textsc{\Large Lab 6 Report }\\[0.5 cm]				% Course Code
	\rule{\linewidth}{0.2 mm} \\[0.4 cm]
	{ \huge \bfseries \thetitle}\\
	\rule{\linewidth}{0.2 mm} \\[1.5 cm]
	
	\begin{minipage}{0.4\textwidth}
		\begin{flushleft} \large
		%	\emph{Submitted To:}\\
		%	Name\\
          % Affiliation\\
           %contact info\\
			\end{flushleft}
			\end{minipage}~
			\begin{minipage}{0.4\textwidth}
            
			\begin{flushright} \large
			\emph{Submitted By :} \\
			Abdalrahman Alhajri 
		\end{flushright}
           
	\end{minipage}\\[2 cm]
	
%	\includegraphics[scale = 0.5]{PICMathLogo.png}
    
    
    
    
	
\end{titlepage}

%%%%%%%%%%%%%%%%%%%%%%%%%%%%%%%%%%%%%%%%%%%%%%%%%%%%%%%%%%%%%%%%%%%%%%%%%%%%%%%%%%%%%%%%%

\tableofcontents
\pagebreak

%%%%%%%%%%%%%%%%%%%%%%%%%%%%%%%%%%%%%%%%%%%%%%%%%%%%%%%%%%%%%%%%%%%%%%%%%%%%%%%%%%%%%%%%%
\renewcommand{\thesection}{\arabic{section}}
\section{Introduction}
The goal of this lab is to become familiar with the scipy.signal.residue() function so that it can be used to perform partial fraction expansion. Before the lab, we were given a differential equation and asked to find its transfer function. Then, we found its the output for a step input using partial fraction expansion. We used our results throughout this lab procedure.

\section{Methodology}

This lab consists of two parts:
\begin{itemize}
    \item
    Part 1:\\
    The first part of this lab focuses on plotting our results from the prelab and comparing it with the output of the scipy.signal.step() function. Therefore, we defined a function with the results from the prelab in the time domain for the step response, then used arrays to define our transfer function $H(s)$. We plotted the two functions and got the identical graphs shown in figure 1.\\ 
    In addition, this part introduces the functionality of the scipy.signal.residue() function. This function takes arrays as inputs that represent the output function $Y(s)=H(s)X(s)$ in the $s$ domain, and return the outputs: residues (R), poles (P), and gains (K). These outputs are used later in the second part of this lab.

    \item
    Part 2:\\
    This part focuses on building the cosine method function, which is described in the textbook as:
    $$f_c(t)=\frac{1}{2}|k| e^{\alpha t}\cos(\omega t + \angle k)u(t)$$
    Where $k$ is represented by the the outcome of the scipy.signal.residue() function as residue R and $\alpha$ and $\omega$ as the components of the poles P. The output of the cosine method function for a fifth order differential equation is plotted in figure 2 and compared to the plot for the scipy.signal.step() function for the response of the differential equation. 

\end{itemize}
   
\section{Results}
\begin{itemize}
    \item Part 1:
    The two plots for the user-defined and built-in functions of the step response for the second order differential equation given in the pre-lab are shown in figure 1 as following:\\
    \includegraphics[]{figure 1.png}\\
    And the partial fraction expansion outcome of the scipy.signal.residue() function for that equation is shown as:\\
    \includegraphics[]{output to task 3.PNG}\\
    
     \item Part 2:
    The partial fraction expansion outcome of the scipy.signal.residue() function applied to the fifth order differential equation given in this part is shown as:\\
    \includegraphics[]{output to task 4.PNG}\\
    And using these results as inputs for the user-defined cosine method function will result for a graph matching the built-in functions of the step response. These two plots are shown in figure 2 as following:\\
    \includegraphics[]{figure 2.png}
    
\end{itemize}


\section{Questions}

\begin{enumerate}
    \item 
    For a non-complex pole-residue term, you can still use the cosine method, explain why this works.\\
    The complex term in the pole-residue represents the frequency of the cosine term. Therefore, for a non-complex pole-residue term, the frequency in the cosine term will be zero, which is going to result for the cosine term to have a constant value, and the overall function will act as an exponential term.
    
    \item
    Leave any feedback on the clarity of the expectations, instructions, and deliverables.\\
    This lab was good, but the hardest part for me was the cosin method. however, every things were good.
    
\end{enumerate}

\section{Conclusion}
Throughout this lab, we built our own user-defined cosine method function using our knowledge about the results of the residue function that we gained throughout the procedure. Therefore, we can use our cosine method along with the scipy.signal.residue function to get the response for any differential equation describing a system's output.

\end{document}

%This template was created by Roza Aceska.