%%%%%%%%%%%%%%%%%%%%%%%%%%%%%%%%%%%%%%%%%%%
%%% DOCUMENT PREAMBLE %%%
\documentclass[12pt]{report}
\usepackage[english]{babel}
%\usepackage{natbib}
\usepackage{url}
\usepackage[utf8x]{inputenc}
\usepackage{amsmath}
\usepackage{graphicx}
\graphicspath{{images/}}
\usepackage{parskip}
\usepackage{fancyhdr}
\usepackage{vmargin}
\usepackage{listings}
\usepackage{hyperref}
\usepackage{xcolor}

\definecolor{codegreen}{rgb}{0,0.6,0}
\definecolor{codegray}{rgb}{0.5,0.5,0.5}
\definecolor{codeblue}{rgb}{0,0,0.95}
\definecolor{backcolour}{rgb}{0.95,0.95,0.92}

\lstdefinestyle{mystyle}{
    backgroundcolor=\color{backcolour},   
    commentstyle=\color{codegreen},
    keywordstyle=\color{codeblue},
    numberstyle=\tiny\color{codegray},
    stringstyle=\color{codegreen},
    basicstyle=\ttfamily\footnotesize,
    breakatwhitespace=false,         
    breaklines=true,                 
    captionpos=b,                    
    keepspaces=true,                 
    numbers=left,                    
    numbersep=5pt,                  
    showspaces=false,                
    showstringspaces=false,
    showtabs=false,                  
    tabsize=2
}
 
\lstset{style=mystyle}

\setmarginsrb{3 cm}{2.5 cm}{3 cm}{2.5 cm}{1 cm}{1.5 cm}{1 cm}{1.5 cm}

\title{Lab 7: Block Diagrams and System Stability}								
% Title
\author{ Abdalrahman Alhajri}						
% Author
\date{Oct 21, 2021}
% Date

\makeatletter
\let\thetitle\@title
\let\theauthor\@author
\let\thedate\@date
\makeatother

\pagestyle{fancy}
\fancyhf{}
\rhead{\theauthor}
\lhead{\thetitle}
\cfoot{\thepage}
%%%%%%%%%%%%%%%%%%%%%%%%%%%%%%%%%%%%%%%%%%%%
\begin{document}

%%%%%%%%%%%%%%%%%%%%%%%%%%%%%%%%%%%%%%%%%%%%%%%%%%%%%%%%%%%%%%%%%%%%%%%%%%%%%%%%%%%%%%%%%

\begin{titlepage}
	\centering
    \vspace*{0.5 cm}
   % \includegraphics[scale = 0.075]{bsulogo.png}\\[1.0 cm]	% University Logo
\begin{center}    \textsc{\Large   ECE 351 - Section \#52 }\\[2.0 cm]	\end{center}% University Name
	\textsc{\Large Lab 7 Report }\\[0.5 cm]				% Course Code
	\rule{\linewidth}{0.2 mm} \\[0.4 cm]
	{ \huge \bfseries \thetitle}\\
	\rule{\linewidth}{0.2 mm} \\[1.5 cm]
	
	\begin{minipage}{0.4\textwidth}
		\begin{flushleft} \large
		%	\emph{Submitted To:}\\
		%	Name\\
          % Affiliation\\
           %contact info\\
			\end{flushleft}
			\end{minipage}~
			\begin{minipage}{0.4\textwidth}
            
			\begin{flushright} \large
			\emph{Submitted By :} \\
			Abdalrahman Alhajri  
		\end{flushright}
           
	\end{minipage}\\[2 cm]
	
%	\includegraphics[scale = 0.5]{PICMathLogo.png}
    
    
    
    
	
\end{titlepage}

%%%%%%%%%%%%%%%%%%%%%%%%%%%%%%%%%%%%%%%%%%%%%%%%%%%%%%%%%%%%%%%%%%%%%%%%%%%%%%%%%%%%%%%%%

\tableofcontents
\pagebreak

%%%%%%%%%%%%%%%%%%%%%%%%%%%%%%%%%%%%%%%%%%%%%%%%%%%%%%%%%%%%%%%%%%%%%%%%%%%%%%%%%%%%%%%%%
\renewcommand{\thesection}{\arabic{section}}
\section{Introduction}
The goal of this lab is to become familiar with Laplace-domain block diagrams and use the partial fraction form of the transfer function to judge a system's stability. Throughout the lab procedure, we did some work by hand and compared it to the results from the code using Python.


\section{Equations}

\begin{itemize}
    \item
    Part 1:\\
    The factored form, poles, and zeros of the three given functions are listed below:\\
    $G(s)=\frac{s+9}{(s-8)(s+2)(s+4)}$\\
    Z: -9\\
    P: 8, -2, -4\\
    
    $A(s)=\frac{s+4}{(s+1)(s+3)}$\\
    Z: -4\\
    P: -1, -3\\
    
    $B(s)=(s+12)(s+14)$\\
    Z: -12, -14
    
    The open-loop transfer function of the block diagram is:
    $$H_o(s)= \frac{Y(s)}{X(s)}=A(s)\cdot G(s)=\frac{(s+9)(s+4)}{(s-8)(s+2)(s+4)(s+1)(s+3)}$$$$=\frac{(s+9)}{(s-8)(s+2)(s+1)(s+3)}$$
    \item
    Part 2:\\
    The closed-loop transfer function of the block diagram is:
    $$H_c(s)= \frac{Y(s)}{X(s)}=\frac{A(s)\cdot G(s)}{1+A(s)\cdot G(s)}=\frac{numA\cdot numG}{denA\cdot denG+denA\cdot numG\cdot B}$$
    $$=\frac{(s+9)(s+4)}{(s+5.16-j9.52)(s+5.16+j9.52)(s+6.175)(s+3)(s+1)}$$

\end{itemize}

\section{Methodology}

This lab consists of two parts:
\begin{itemize}
    \item
    Part 1:\\
    The first part of this lab focuses on analyzing the open-loop function of the given block diagram. To do so, we factored the given equations, $G(s)$, $A(s)$, and $B(s)$, and identified their zeros and poles as shown in the equations section. Then we used the built-in function scipy.signal.tf2zpk() function to check our results. The output is shown in the results section. Then, by looking at the block diagram we identified the open-loop transfer function $H_o(s)$ as shown in the equations section. After that, we plotted the step response for the open-loop transfer function as shown in the results section.

    \item
    Part 2:\\
    This part carries on the process of learning how to analyze a block diagram. However, we will be focusing on the closed-loop instead. In order to do that, we symbolically derived the transfer function $H_c(s)$ as shown in the equations section using the $numG$ and $denG$ referring to the numerator and denominator of the function $G(s)$. Then we used that along with the function scipy.signal.tf2zpk() to find the zeros and poles for the transfer function of the closed-loop. We, then, plotted the step response for the closed-loop transfer function using scipy.signal.step().

\end{itemize}
   
\section{Results}
\begin{itemize}
    \item Part 1:\\
    The output of the function scipy.signal.tf2zpk() applied to the three given equations to verify their zeros and poles is shown below:\\
    \includegraphics[]{output to task 2.PNG}\\
    By looking at the poles for the transfer function for the open-loop $H_o(s)$ (or its factored denominator) as shown in the equation section, we see that the poles are once positive and others negative. This indicate that the system is unstable. and the graph in figure 1 for the step response shown below agrees with this conclusion where we see that as time goes to infinity, our output diverges.\\
    \includegraphics[]{figure 1.png}
    
    \item Part 2:\\
    By looking at the numerical result for the transfer function of the closed-loop $H_c(s)$ we can see that the poles agree on the sign. Therefore, $H_c(s)$ is stable. In other words, all the time domain exponential terms have negative power. Figure 2 shows that this transfer function is actually stable. We see that the plot for the step response converges as we take the limit to infinity.\\
    \includegraphics[]{figure 2.png}
    
\end{itemize}


\section{Questions}

\begin{enumerate}
    \item 
    In Part 1 Task 5, why does convolving the factored terms using scipy.signal.convolve() result in the expanded form of the numerator and denominator? Would this work with your user-defined convolution function from Lab 3? Why or why not?\\
    Since we are in the frequency domain $s$, the built-in function scipy.signal.convolve() will realize that, where the user-defined convolution function we built in lab 3 was only designed for the time domain t. Therefore, it won't work that way.
    
    \item
    Discuss the difference between the open- and closed-loop systems from Part 1 and Part 2. How does stability differ for each case, and why?\\
    The open-loop for the system is unstable due to the characteristics of its function $G(s)$. However, the closed-loop represented by a negative feedback loop is designed to stabilize the system. 
    
    \item
    What is the difference between scipy.signal.residue() used in Lab 6 and scipy.signal.tf2zpk() used in this lab?\\
    The scipy.signal.residue() function is used to compute the partial fraction expansion of a transfer function, where the scipy.signal.tf2zpk() is used to find the zeros, poles, and gain for a numerator and denominator representing a function.
    
    \item
    Is it possible for an open-loop system to be stable? What about for a closed-loop system to be unstable? Explain how or how not for each.\\
    We can check stability for both open loop or closed loop systems. The fact that the system is open or closed loop does not have a direct bearing on its stability. In other words, both can be stable or unstable. Feedback can be used to stabilize an unstable system (i.e., an open loop unstable system can become stable under closed loop such as the one in the system given in this lab) but bear in mind that, without proper care, it is also possible to destabilize open loop stable systems with a closed loop.
    

    \item
    Leave any feedback on the clarity of the expectations, instructions, and deliverables.\\
    I think the lab was clear.
\end{enumerate}

\section{Conclusion}
Throughout this lab, we gained a new skill which is analyzing block diagrams using the transfer function of an open-loop and a closed-loop. We also learned how to identify a system's stability graphically and numerically. 

\end{document}

%This template was created by Roza Aceska.
