%%%%%%%%%%%%%%%%%%%%%%%%%%%%%%%%%%%%%%%%%%%
%%% DOCUMENT PREAMBLE %%%
\documentclass[12pt]{report}
\usepackage[english]{babel}
%\usepackage{natbib}
\usepackage{url}
\usepackage[utf8x]{inputenc}
\usepackage{amsmath}
\usepackage{graphicx}
\graphicspath{{images/}}
\usepackage{parskip}
\usepackage{fancyhdr}
\usepackage{vmargin}
\usepackage{listings}
\usepackage{hyperref}
\usepackage{xcolor}

\definecolor{codegreen}{rgb}{0,0.6,0}
\definecolor{codegray}{rgb}{0.5,0.5,0.5}
\definecolor{codeblue}{rgb}{0,0,0.95}
\definecolor{backcolour}{rgb}{0.95,0.95,0.92}

\lstdefinestyle{mystyle}{
    backgroundcolor=\color{backcolour},   
    commentstyle=\color{codegreen},
    keywordstyle=\color{codeblue},
    numberstyle=\tiny\color{codegray},
    stringstyle=\color{codegreen},
    basicstyle=\ttfamily\footnotesize,
    breakatwhitespace=false,         
    breaklines=true,                 
    captionpos=b,                    
    keepspaces=true,                 
    numbers=left,                    
    numbersep=5pt,                  
    showspaces=false,                
    showstringspaces=false,
    showtabs=false,                  
    tabsize=2
}
 
\lstset{style=mystyle}

\setmarginsrb{3 cm}{2.5 cm}{3 cm}{2.5 cm}{1 cm}{1.5 cm}{1 cm}{1.5 cm}

\title{Lab 4: System Step Response Using Convolution }								
% Title
\author{ Abdalrahman Alhajri}						
% Author
\date{Sep 28 ,2021}
% Date

\makeatletter
\let\thetitle\@title
\let\theauthor\@author
\let\thedate\@date
\makeatother

\pagestyle{fancy}
\fancyhf{}
\rhead{\theauthor}
\lhead{\thetitle}
\cfoot{\thepage}
%%%%%%%%%%%%%%%%%%%%%%%%%%%%%%%%%%%%%%%%%%%%
\begin{document}

%%%%%%%%%%%%%%%%%%%%%%%%%%%%%%%%%%%%%%%%%%%%%%%%%%%%%%%%%%%%%%%%%%%%%%%%%%%%%%%%%%%%%%%%%

\begin{titlepage}
	\centering
    \vspace*{0.5 cm}
   % \includegraphics[scale = 0.075]{bsulogo.png}\\[1.0 cm]	% University Logo
\begin{center}    \textsc{\Large   ECE 351 - Section \#52 }\\[2.0 cm]	\end{center}% University Name
	\textsc{\Large Lab 4 Report }\\[0.5 cm]				% Course Code
	\rule{\linewidth}{0.2 mm} \\[0.4 cm]
	{ \huge \bfseries \thetitle}\\
	\rule{\linewidth}{0.2 mm} \\[1.5 cm]
	
	\begin{minipage}{0.4\textwidth}
		\begin{flushleft} \large
		%	\emph{Submitted To:}\\
		%	Name\\
          % Affiliation\\
           %contact info\\
			\end{flushleft}
			\end{minipage}~
			\begin{minipage}{0.4\textwidth}
            
			\begin{flushright} \large
			\emph{Submitted By :} \\
			Abdalrahman Alhajri  
		\end{flushright}
           
	\end{minipage}\\[2 cm]
	
%	\includegraphics[scale = 0.5]{PICMathLogo.png}
    
    
    
    
	
\end{titlepage}

%%%%%%%%%%%%%%%%%%%%%%%%%%%%%%%%%%%%%%%%%%%%%%%%%%%%%%%%%%%%%%%%%%%%%%%%%%%%%%%%%%%%%%%%%

\tableofcontents
\pagebreak

%%%%%%%%%%%%%%%%%%%%%%%%%%%%%%%%%%%%%%%%%%%%%%%%%%%%%%%%%%%%%%%%%%%%%%%%%%%%%%%%%%%%%%%%%
\renewcommand{\thesection}{\arabic{section}}
\section{Introduction}
In this lab, we will be using our convolution function we built last lab in order to compute a system's step response. In addition, we will practice indexing and rearranging the way we call our convolution function so that it fits our system.

\section{Equations}

For this lab, the functions h1(t), h2(t), and h3(t) were used to be plotted:
\begin{equation}
    h1(t) =  e^{−2t}[u(t) − u(t − 3)]
\end{equation}

\begin{equation}
    h2(t) = u(t − 2) − u(t − 6)
\end{equation}

\begin{equation}
    h3(t) = cos(wt)u(t)
\end{equation}

Then we computed, by hand, their step response by convolving them with the step function and we got the following functions:
\begin{equation}
    h1(t)*u(t) = \frac{1}{2}*(1 - e(-2*t))*u(t) + (1/2)*(e(-2*t) - e(-6))*u(t-3)
\end{equation}
\begin{equation}
    h2(t)*u(t) = (t-2)u(t-2) - (t-6)u(t-6)
\end{equation}
\begin{equation}
    h3(t)*u(t) = \frac{1}{w}sin(wt)u(t)
\end{equation}

\section{Methodology}

This lab consists of two parts:
\begin{itemize}
    \item
    Part 1:
    
    This part was simply focused on building the three transfer functions, shown in the Equations section, using our step function created in lab 2. We, then, plotted the three functions as shown in figure 1.

    \item
    Part 2:\\
    in this part, we applied our convolution function to convolve the three transfer functions individually with the our step function. In order to do that, we had to rearrange the range of the variable "t" so that the convolution graph works appropriately for us. After that, we found the step response for the functions by convolving them with the step function by hand. Then we compared the two graphs and got the results as shown in figure 2 and figure 3.
\end{itemize}
   
\section{Results}
\begin{itemize}
    \item Part 1:
    The three user-defined transfer functions, h1(t), h2(t), and h3(t), are plotted individually in figure 1 as following:\\
    \includegraphics[]{figure1.png}\\
    
     \item Part 2:
    Using the convolution function developed last lab, we convolved the transfer functions with the step functions to get the figure 2 as follows\\
    \includegraphics[]{figure2.png}\\
    
    Then, by hand we worked the three convolutions and got the three functions shown in the Equations section and plotted them in figure 3 as shown:\\
    \includegraphics[]{figure3.png}\\
    
\end{itemize}

\section{Questions}

\begin{enumerate}
    \item 
     Leave any feedback on the clarity of lab tasks, expectations, and deliverables.\\
     
     The procedure and outline of the lab is very clear.  
\end{enumerate}

\section{Conclusion}

Throughout this lab, we applied our own user-defined convolution function on given functions convolved with the step function to get the step response of each. In the process of that, we learned how to set up the indexing of the input range so the code function the way it should.

\end{document}

%This template was created by Roza Aceska.