%%%%%%%%%%%%%%%%%%%%%%%%%%%%%%%%%%%%%%%%%%%
%%% DOCUMENT PREAMBLE %%%
\documentclass[12pt]{report}
\usepackage[english]{babel}
%\usepackage{natbib}
\usepackage{url}
\usepackage[utf8x]{inputenc}
\usepackage{amsmath}
\usepackage{graphicx}
\graphicspath{{images/}}
\usepackage{parskip}
\usepackage{fancyhdr}
\usepackage{vmargin}
\usepackage{listings}
\usepackage{hyperref}
\usepackage{xcolor}

\definecolor{codegreen}{rgb}{0,0.6,0}
\definecolor{codegray}{rgb}{0.5,0.5,0.5}
\definecolor{codeblue}{rgb}{0,0,0.95}
\definecolor{backcolour}{rgb}{0.95,0.95,0.92}

\lstdefinestyle{mystyle}{
    backgroundcolor=\color{backcolour},   
    commentstyle=\color{codegreen},
    keywordstyle=\color{codeblue},
    numberstyle=\tiny\color{codegray},
    stringstyle=\color{codegreen},
    basicstyle=\ttfamily\footnotesize,
    breakatwhitespace=false,         
    breaklines=true,                 
    captionpos=b,                    
    keepspaces=true,                 
    numbers=left,                    
    numbersep=5pt,                  
    showspaces=false,                
    showstringspaces=false,
    showtabs=false,                  
    tabsize=2
}
 
\lstset{style=mystyle}

\setmarginsrb{3 cm}{2.5 cm}{3 cm}{2.5 cm}{1 cm}{1.5 cm}{1 cm}{1.5 cm}

\title{Lab 8: Fourier Series Approximation of a Square Wave}								
% Title
\author{Abdalrahman Alhajri}						
% Author
\date{Oct 26,2021}
% Date

\makeatletter
\let\thetitle\@title
\let\theauthor\@author
\let\thedate\@date
\makeatother

\pagestyle{fancy}
\fancyhf{}
\rhead{\theauthor}
\lhead{\thetitle}
\cfoot{\thepage}
%%%%%%%%%%%%%%%%%%%%%%%%%%%%%%%%%%%%%%%%%%%%
\begin{document}

%%%%%%%%%%%%%%%%%%%%%%%%%%%%%%%%%%%%%%%%%%%%%%%%%%%%%%%%%%%%%%%%%%%%%%%%%%%%%%%%%%%%%%%%%

\begin{titlepage}
	\centering
    \vspace*{0.5 cm}
   % \includegraphics[scale = 0.075]{bsulogo.png}\\[1.0 cm]	% University Logo
\begin{center}    \textsc{\Large   ECE 351 - Section \#52 }\\[2.0 cm]	\end{center}% University Name
	\textsc{\Large Lab 8 Report }\\[0.5 cm]				% Course Code
	\rule{\linewidth}{0.2 mm} \\[0.4 cm]
	{ \huge \bfseries \thetitle}\\
	\rule{\linewidth}{0.2 mm} \\[1.5 cm]
	
	\begin{minipage}{0.4\textwidth}
		\begin{flushleft} \large
		%	\emph{Submitted To:}\\
		%	Name\\
          % Affiliation\\
           %contact info\\
			\end{flushleft}
			\end{minipage}~
			\begin{minipage}{0.4\textwidth}
            
			\begin{flushright} \large
			\emph{Submitted By :} \\
			Abdalrahman Alhajri 
		\end{flushright}
           
	\end{minipage}\\[2 cm]
	
%	\includegraphics[scale = 0.5]{PICMathLogo.png}
    
    
    
    
	
\end{titlepage}

%%%%%%%%%%%%%%%%%%%%%%%%%%%%%%%%%%%%%%%%%%%%%%%%%%%%%%%%%%%%%%%%%%%%%%%%%%%%%%%%%%%%%%%%%

\tableofcontents
\pagebreak

%%%%%%%%%%%%%%%%%%%%%%%%%%%%%%%%%%%%%%%%%%%%%%%%%%%%%%%%%%%%%%%%%%%%%%%%%%%%%%%%%%%%%%%%%
\renewcommand{\thesection}{\arabic{section}}
\section{Introduction}
The goal of this lab is to use Fourier Series to approximate periodic time-domain signals. Before the lab, we were given the following square wave function and were asked to find its a Fourier series. \\
\includegraphics[]{figure.PNG}


\section{Equations}
The Fourier series for the wave given is shown below and it is used to show the output and plots shown in the results section.
$$x(t) = \sum_{k=1}^{\infty}[\frac{1}{k\pi}( 1-\cos{(2k\pi)}) \sin{(k\omega_0 t)}]$$

\section{Methodology}
This lab procedure is very short and straightforward. first we put the $a_k$ term and $b_k$ into Python and printed their first 3 terms. Then we created a Fourier series using for loops and plotted 6 graphs with $N = {1, 3, 15, 50, 150, 1500}$ terms.
   
\section{Results}
The output shown below is the first three terms of $a_k$ and $b_k$. \\
\includegraphics[]{output to task 1.PNG}\\
And since $a_k$ is zero for all $k$ we didn't have to show all it's terms and we didn't use it in order to create the Fourier series for the given function.

The plots below show the Fourier series approximations with $N$ numbers of terms.\\
\includegraphics[]{figure 1.png}\\
\includegraphics[]{figure 2.png}\\
We see that as we start including more numbers terms we get closer to the square wave function we were given at the beginning.

\section{Questions}

\begin{enumerate}
    \item 
    Is x(t) an even or an odd function? Explain why.\\
    It is an odd function because 
    $$x(-t)=-x(t)$$
    
    \item
    Based on your results from Task 1, what do you expect the values of a2, a3, . . . , an to be? Why?\\
    All zeros because $x(t)$ is an odd function, so we expect the coefficient of the cosine term to be zero.
    
    \item
    How does the approximation of the square wave change as the value of N increases? In what way does the Fourier series struggle to approximate the square wave?\\
    The more terms we include (higher value for $N$), the more accurate our approximation is. This is due to higher number of sinusoidal terms summed together with different amplitudes. And the points where the approximation struggles the most is the amplitudes once it changes signs. I believe this is due to the amplitude of the first term of the Fourier series.
    
    \item
    What is occurring mathematically in the Fourier series summation as the value of N increases?\\
    As we use more terms, the amplitudes of the terms start getting closer to each other so we see that the wave start to soften itself to create a square wave instead of a sinusoidal one. 
    \item
    Leave any feedback on the clarity of the expectations, instructions, and deliverables.\\
    I think this lab is very clear and straightforward.
    
\end{enumerate}

\section{Conclusion}
Throughout this lab, we gained knowledge on how to find Fourier series for a given wave function using a different method to the one used in the lecture. Along with that, we walked through the difference between odd and even Fourier series graphically and mathematically. 

\end{document}

%This template was created by Roza Aceska.
