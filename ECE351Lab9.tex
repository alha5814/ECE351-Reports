%%%%%%%%%%%%%%%%%%%%%%%%%%%%%%%%%%%%%%%%%%%
%%% DOCUMENT PREAMBLE %%%
\documentclass[12pt]{report}
\usepackage[english]{babel}
%\usepackage{natbib}
\usepackage{url}
\usepackage[utf8x]{inputenc}
\usepackage{amsmath}
\usepackage{graphicx}
\graphicspath{{images/}}
\usepackage{parskip}
\usepackage{fancyhdr}
\usepackage{vmargin}
\usepackage{listings}
\usepackage{hyperref}
\usepackage{xcolor}

\definecolor{codegreen}{rgb}{0,0.6,0}
\definecolor{codegray}{rgb}{0.5,0.5,0.5}
\definecolor{codeblue}{rgb}{0,0,0.95}
\definecolor{backcolour}{rgb}{0.95,0.95,0.92}

\lstdefinestyle{mystyle}{
    backgroundcolor=\color{backcolour},   
    commentstyle=\color{codegreen},
    keywordstyle=\color{codeblue},
    numberstyle=\tiny\color{codegray},
    stringstyle=\color{codegreen},
    basicstyle=\ttfamily\footnotesize,
    breakatwhitespace=false,         
    breaklines=true,                 
    captionpos=b,                    
    keepspaces=true,                 
    numbers=left,                    
    numbersep=5pt,                  
    showspaces=false,                
    showstringspaces=false,
    showtabs=false,                  
    tabsize=2
}
 
\lstset{style=mystyle}

\setmarginsrb{3 cm}{2.5 cm}{3 cm}{2.5 cm}{1 cm}{1.5 cm}{1 cm}{1.5 cm}

\title{Lab 9: Fast Fourier Transform}								
% Title
\author{Abdalrahman Alhajri}						
% Author
\date{Nov, 3, 2021}
% Date

\makeatletter
\let\thetitle\@title
\let\theauthor\@author
\let\thedate\@date
\makeatother

\pagestyle{fancy}
\fancyhf{}
\rhead{\theauthor}
\lhead{\thetitle}
\cfoot{\thepage}
%%%%%%%%%%%%%%%%%%%%%%%%%%%%%%%%%%%%%%%%%%%%
\begin{document}

%%%%%%%%%%%%%%%%%%%%%%%%%%%%%%%%%%%%%%%%%%%%%%%%%%%%%%%%%%%%%%%%%%%%%%%%%%%%%%%%%%%%%%%%%

\begin{titlepage}
	\centering
    \vspace*{0.5 cm}
   % \includegraphics[scale = 0.075]{bsulogo.png}\\[1.0 cm]	% University Logo
\begin{center}    \textsc{\Large   ECE 351 - Section \#52 }\\[2.0 cm]	\end{center}% University Name
	\textsc{\Large Lab 9 Report }\\[0.5 cm]				% Course Code
	\rule{\linewidth}{0.2 mm} \\[0.4 cm]
	{ \huge \bfseries \thetitle}\\
	\rule{\linewidth}{0.2 mm} \\[1.5 cm]
	
	\begin{minipage}{0.4\textwidth}
		\begin{flushleft} \large
		%	\emph{Submitted To:}\\
		%	Name\\
          % Affiliation\\
           %contact info\\
			\end{flushleft}
			\end{minipage}~
			\begin{minipage}{0.4\textwidth}
            
			\begin{flushright} \large
			\emph{Submitted By :} \\
			Abdalrahman Alhajri
		\end{flushright}
           
	\end{minipage}\\[2 cm]
	
%	\includegraphics[scale = 0.5]{PICMathLogo.png}
    
    
    
    
	
\end{titlepage}

%%%%%%%%%%%%%%%%%%%%%%%%%%%%%%%%%%%%%%%%%%%%%%%%%%%%%%%%%%%%%%%%%%%%%%%%%%%%%%%%%%%%%%%%%

\tableofcontents
\pagebreak

%%%%%%%%%%%%%%%%%%%%%%%%%%%%%%%%%%%%%%%%%%%%%%%%%%%%%%%%%%%%%%%%%%%%%%%%%%%%%%%%%%%%%%%%%
\renewcommand{\thesection}{\arabic{section}}
\section{Introduction}
The objective of this lab is to become familiar with fast Fourier transforms using Python. To do so, we will create our own Fast Fourier Transform (fft) routine as a user-defined function and analyze and plot some given sinusoidal functions.

\section{Equations}
Throughout the lab, we plotted four different functions. These functions are given below with their initialization corresponding to the numbers of their figures in the results section.

$$x_1(t) = \cos{(2\pi t)}$$

$$x_2(t) = 5\sin{(2\pi t)}$$

$$x_3(t) = 2\cos{((2\pi \cdot 2t)-2)} + \sin^2{((2\pi \cdot 6t)+3)}$$

$$x_4(t) = \sum_{k=1}^{15}[\frac{1}{k\pi}( 1-\cos{(2k\pi)}) \sin{(\frac{k\pi t}{4})}]$$

\section{Methodology}
This lab procedure is very short and straightforward. First, we created our own user-defined function for the Fast Fourier Transform (fft) using the code provided by the instructor. The given code was simply showing how to calculate the magnitude and phase shift of an input signal and how to plot them. To do the lab, we fixed that code so it involves a user-defined fft function.\\
We then defined the given signals as inputs and plotted them individually with their magnitude and phase plots as shown in the results section. At the end, we plotted the Fourier series,$x_4(t)$, implemented in lab 8 with its magnitude and phase as shown in figure 4.
   
\section{Results}
The figures below shows the first three signals before editing the code:\\
\includegraphics[]{Figure 1(1).png}\\
\includegraphics[]{Figure 2(1).png}\\
\includegraphics[]{Figure 3(1).png}\\

The figures below correspond to the four given signals. They graphically describe the functions in the time domain, the magnitudes in the frequency domain, and their phase shift.\\
\includegraphics[]{Figure 1.png}\\
\includegraphics[]{Figure 2.png}\\
\includegraphics[]{Figure 3.png}\\
\includegraphics[]{Figure 4.png}\\
Here we see that the phases are only showing where there are magnitudes that actually count, unlike the first three graphs at the top.

\section{Questions}

\begin{enumerate}
    \item 
    What happens if fs is lower? If it is higher? fs in your report must span a few orders of magnitude.\\
    Using lower fs will result in larger T as $T=\frac{1}{fs}$, which will give us less points to plot in the time domain and a smaller range in the frequency domain. On the other hand, using larger fs will result in smaller T as $T=\frac{1}{fs}$, which will give us more points to plot in the time domain and a larger range in the frequency domain.\\
    
    \item
    What difference does eliminating the small phase magnitudes make?\\
    It allow us to focus only on the phases of frequencies that correspond to magnitudes large enough to be considered and eliminate the rest.\\
    
    \item
    Verify your results from Tasks 1 and 2 using the Fourier transforms of cosine and sine. Explain why your results are correct.\\
    $$x_1(t) = \cos{(2\pi t)}$$
    $$X_1(f) = \frac{1}{2}[\delta(f+1) + \delta(f-1)]$$
    Here, we see that the magnitude should be $+0.5$ at $f=-1,1$. 
    
    $$x_2(t) = 5\sin{(2\pi t)}$$
    $$X_2(f) = \frac{j5}{2}[\delta(f+1) - \delta(f-1)]$$
    Here, we see that the magnitude should be $+2.5$ at $f=-1,1$\\
    
    \item
    Leave any feedback on the clarity of the expectations, instructions, and deliverables.\\
   This lab took a lot of time from me to work on it, but it was good. 
\end{enumerate}

\section{Conclusion}
Throughout this lab, we gained knowledge on how to find Fast Fourier Transform(fft) for a given signal using the fft pack. Along with that, we practiced new and different plotting skills in Python such as using plt.stem and organizing figures with several plots.

\end{document}

%This template was created by Roza Aceska.