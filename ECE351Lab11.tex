%%%%%%%%%%%%%%%%%%%%%%%%%%%%%%%%%%%%%%%%%%%
%%% DOCUMENT PREAMBLE %%%
\documentclass[12pt]{report}
\usepackage[english]{babel}
%\usepackage{natbib}
\usepackage{url}
\usepackage[utf8x]{inputenc}
\usepackage{amsmath}
\usepackage{graphicx}
\graphicspath{{images/}}
\usepackage{parskip}
\usepackage{fancyhdr}
\usepackage{vmargin}
\usepackage{listings}
\usepackage{hyperref}
\usepackage{xcolor}

\definecolor{codegreen}{rgb}{0,0.6,0}
\definecolor{codegray}{rgb}{0.5,0.5,0.5}
\definecolor{codeblue}{rgb}{0,0,0.95}
\definecolor{backcolour}{rgb}{0.95,0.95,0.92}

\lstdefinestyle{mystyle}{
    backgroundcolor=\color{backcolour},   
    commentstyle=\color{codegreen},
    keywordstyle=\color{codeblue},
    numberstyle=\tiny\color{codegray},
    stringstyle=\color{codegreen},
    basicstyle=\ttfamily\footnotesize,
    breakatwhitespace=false,         
    breaklines=true,                 
    captionpos=b,                    
    keepspaces=true,                 
    numbers=left,                    
    numbersep=5pt,                  
    showspaces=false,                
    showstringspaces=false,
    showtabs=false,                  
    tabsize=2
}
 
\lstset{style=mystyle}

\setmarginsrb{3 cm}{2.5 cm}{3 cm}{2.5 cm}{1 cm}{1.5 cm}{1 cm}{1.5 cm}

\title{Lab 11: Z - Transform Operations}								
% Title
\author{Abdalrahman Alhajri}						
% Author
\date{Nov 17, 2021}
% Date

\makeatletter
\let\thetitle\@title
\let\theauthor\@author
\let\thedate\@date
\makeatother

\pagestyle{fancy}
\fancyhf{}
\rhead{\theauthor}
\lhead{\thetitle}
\cfoot{\thepage}
%%%%%%%%%%%%%%%%%%%%%%%%%%%%%%%%%%%%%%%%%%%%
\begin{document}

%%%%%%%%%%%%%%%%%%%%%%%%%%%%%%%%%%%%%%%%%%%%%%%%%%%%%%%%%%%%%%%%%%%%%%%%%%%%%%%%%%%%%%%%%

\begin{titlepage}
	\centering
    \vspace*{0.5 cm}
   % \includegraphics[scale = 0.075]{bsulogo.png}\\[1.0 cm]	% University Logo
\begin{center}    \textsc{\Large   ECE 351 - Section \#52 }\\[2.0 cm]	\end{center}% University Name
	\textsc{\Large Lab 11 Report }\\[0.5 cm]				% Course Code
	\rule{\linewidth}{0.2 mm} \\[0.4 cm]
	{ \huge \bfseries \thetitle}\\
	\rule{\linewidth}{0.2 mm} \\[1.5 cm]
	
	\begin{minipage}{0.4\textwidth}
		\begin{flushleft} \large
		%	\emph{Submitted To:}\\
		%	Name\\
          % Affiliation\\
           %contact info\\
			\end{flushleft}
			\end{minipage}~
			\begin{minipage}{0.4\textwidth}
            
			\begin{flushright} \large
			\emph{Submitted By :} \\
			Abdalrahman Alhajri 
		\end{flushright}
           
	\end{minipage}\\[2 cm]
	
%	\includegraphics[scale = 0.5]{PICMathLogo.png}
    
    
    
    
	
\end{titlepage}

%%%%%%%%%%%%%%%%%%%%%%%%%%%%%%%%%%%%%%%%%%%%%%%%%%%%%%%%%%%%%%%%%%%%%%%%%%%%%%%%%%%%%%%%%

\tableofcontents
\pagebreak

%%%%%%%%%%%%%%%%%%%%%%%%%%%%%%%%%%%%%%%%%%%%%%%%%%%%%%%%%%%%%%%%%%%%%%%%%%%%%%%%%%%%%%%%%
\renewcommand{\thesection}{\arabic{section}}
\section{Introduction}
The objective of this lab is to analyze a given discrete system using Python’s built-in functions and a function developed by Christopher Felton. The system we are using in this lab is represented by the following causal function:\\
\includegraphics[]{Capture.PNG}

\section{Equations}
The response function $H(z)$ for the causal function is shown as follows:\\
$$H(z)=\frac{Y(z)}{X(z)}=\frac{2z(z-20)}{z^2-10z+16}$$
And by applying partial fraction expansion we get that:\\
$$H(z)=\frac{6z}{z-2}-\frac{4z}{z-8}$$
Therefore, the taking the inverse z-transformation will result in the following function:\\
$$h[k]=[6(2)^k-4(8)^k]u[k]$$


\section{Methodology}
In order to analyze the given system, we needed to find its response function $h[k]$. To do that, we found the z-transform of the causal function and solved for $H(z)=\frac{Y(z)}{X(z)}$. We then divided both sides by $z$ to use it after applying the partial fraction expansion. After that, we multiplied back the $z$ in both sides and found $h[k]$ by taking the inverse z-transformation of $H(z)$. These steps are mathematically shown below:\\
$$y[k]=2x[k]-40x[k-1]+10y[k-1]-16y[k-2]$$
$$Y(z)(1-10z^{-1}+16z^{-2})=X(z)(2-40z^{-1})$$
$$H(z)=\frac{Y(z)}{X(z)}=\frac{2z(z-20)}{z^2-10z+16}$$
$$\frac{H(z)}{z}=\frac{2(z-20)}{z^2-10z+16}$$
$$\frac{H(z)}{z}=\frac{6}{z-2}-\frac{4}{z-8}$$
$$H(z)=\frac{6z}{z-2}-\frac{4z}{z-8}$$
$$h[k]=[6(2)^k-4(8)^k]u[k]$$

To double check our results for the response function, we used the Python function scipy.signal.residuez() to find the poles and residues for the partial fraction expansion of $H(z)$. The output of this step is shown in the results section.\\

To further our analysis, we used the zplane() function, provided by our TA, to obtain the pole-zero plot for $H(z)$. This plot is shown in Figure 1 under the results section.\\
In addition, we plotted the magnitude and phase for the response function $H(z)$ in Figure 2 using the the built-in function scipy.signal.freqz().

\section{Results}
The residues and poles for the response function $H(z)$ is shown below as:\\
\includegraphics[]{output to task 3.PNG}\\
As we see, this matches the result we gained by taking the partial fraction expansion for $H(z)$. We also see that the absolute value of the poles are greater than 1, $|p|>1$. Therefore, we establish that this system is not stable. To further prove that, we obtained the pole-zero plot for $H(z)$ as:\\
\includegraphics[]{Figure 1.png}\\
We can see that the poles for the response function lie outside of the unit circle, which indicates that the system is unstable.\\

The magnitude and phase responses of $H(z)$ are plotted in figure 2 as shown below:\\
\includegraphics[]{Figure 2.png}\\

\section{Questions}

\begin{enumerate}
    \item 
    Looking at the plot generated in Task 4, is H(z) stable? Explain why or why not.\\
    As discussed in the result section, we can see in figure 1 that the poles for the response function lie outside of the unit circle, which indicates that the system is unstable. This is because the values will not be decaying as $|p|>1$.\\
    
    \item
    Leave any feedback on the clarity of the expectations, instructions, and deliverables.\\
   This lab was clear and good.
    
\end{enumerate}

\section{Conclusion}
Throughout this lab, we analyzed a discrete system mathematically and graphically using our knowledge gained in the class lectures about z-transformation along with the Python tools provided in this lab.

\end{document}

%This template was created by Roza Aceska.